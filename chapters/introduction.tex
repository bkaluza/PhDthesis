\chapter{Introduction}

This Chapter explains how to provide citations, publications related to the dissertation and abbreviations throughout the thesis. 

%The thesis starts with a short overview of the area, positioning of the thesis, the main goals and the hypothesis of the thesis. 
\section{Citations}
In order to use the correct bibliography style, the bibliography style \textit{mps4$\_$5} is included in the main file \textit{thesis.tex}. Several examples for citations are provided in continuation.

\begin{enumerate}
	\item Article citation: \citep{Saaty2003a}, \citep{Saaty2003}
	\item Web page citation: \citep{TheEconomist2010}
	\item Author citation: \cite{Zopounidis2006}
	\item Book citation: \citep{BohanecDEXi2011}
	\item Conference article citation: \citep{Baracskai2003}
	\item Several citations: 
	\subitem \citep{BohanecDEXi2011, Burstein2008, Power2002}
	\subitem  \citep{Skinner1999, Ronald} 
	\subitem   \citep{Triantaphyllou, French, Bouyssou2006}
	\subitem \citep{Figueira2005}
	\subitem \citep{Jacquet-Lagreze1982} 
	\subitem \citep{Saaty2008}
	\subitem \citep{Moshkovich1995}
	\subitem \citep{GrecoRSMCDA}
	\subitem \citep{Adam2008, Figueira2005, Bouyssou2006}
	\subitem \citep{Menzies2006, Saaty2008, Zadeh1975, Guo2009, Barron1996} 
\end{enumerate}

\section{Publications related to the dissertation}
Publications related to the dissertation should be entered in  \textit{myPublication.bib} file. In order to enter them in Chapter~\ref{sec:publications}, one should cite (include) them in the file \textit{my$\_$publications.tex}.

In order a publication to appear in the chapter \emph{Publications related to the dissertation}, after changing the file \textit{my$\_$publications.tex}, one has to run \textit{bibtex bu.aux} for all bu*.aux files (bu1.aux, bu2.aux etc.). After compiling, the references to the publications will appear.


\section{Abbreviations}
The first occurrence of an abbreviation has to be followed with its long explanation. See the examples below for different types of abbreviations.

To add a new abbreviation, in the file \emph{abbreviation.txt} use the following command:
\begin{verbatim} 
   \newacronym\<label>}{<abbrv>}{<full>}
\end{verbatim}

In order to make the abbreviations appear in the list of abbreviations, the following procedure should be applied:
\begin{enumerate}
	\item {Apply latex compile twice (from the editor or in command line by using the command latex thesis.txt}
	\item{Run the following two commands in command line:}
		\subitem {makeindex -s thesis.ist -t thesis.alg -o thesis.acr thesis.acn}
		\subitem {makeindex -s thesis.ist -t thesis.glg -o thesis.gls thesis.glo}
\end{enumerate}
%or simply run sh mkGloss.sh in the command line.

Afterwards the latex compile will include the list of abbreviations. Any content changes in the file \emph{abbreviation.txt}, require repeating the above procedure in order for changes to take effect.

The first occurrence of an abbreviation has to be followed with its long explanation. Several examples of usage of long abbreviations are given in continuation.

\begin{itemize}
	\item Adding an abbreviation and a citation: \gls{qq} \citep{BohanecBook}. \gls{qq} is an abbreviation that has been already included. 
	\item \Gls{FNAC} is an abbreviation at the beginning of the sentence.
	\item Adding a new abbreviation in text for \gls{icf}.
	\item Usage of the long plural form of the acronym: \acrlongpl{icf}. 
	\item Usage of the long singular form of the acronym: \acrlong{icf}.
	\item Usage of the short plural form of the abbreviation: \acrshortpl{icf}. 
	\item Usage of the short singular form of the abbreviation: \acrshort{icf}.
\end{itemize}


\section{Contribution}

This is a new section.


\section{Organization of the thesis}

This is a new section.
